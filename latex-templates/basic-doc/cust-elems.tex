%%%%%%%%%%%%%%%%%%%%%%%%%%%%%%%%%%%%%%%%%%%%%%%%%%%%%%%%%%%%%%%%%%%%%%%%%%%%%%%%
% Basic elements
% Version: 1.0 (2019-04-29)
%
% Author: Grigory Okhmak (ohmak88@yandex.ru)
% License:
% CC BY-NC-SA 3.0 (http://creativecommons.org/licenses/by-nc-sa/3.0/)
%
%%%%%%%%%%%%%%%%%%%%%%%%%%%%%%%%%%%%%%%%%%%%%%%%%%%%%%%%%%%%%%%%%%%%%%%%%%%%%%%%

%-------------------------------------------------------------------------------
%	PACKAGES AND OTHER DOCUMENT CONFIGURATIONS
%-------------------------------------------------------------------------------

\usepackage{amsmath,amsfonts,stmaryrd,amssymb} % Math packages
\usepackage{enumerate} % Custom item numbers for enumerations
\usepackage[ruled]{algorithm2e} % Algorithms
\usepackage[framemethod=tikz]{mdframed} % Allows defining custom boxed/framed environments
\usepackage{listings} % File listings, with syntax highlighting
\lstset{
	basicstyle=\ttfamily, % Typeset listings in monospace font
}

%-------------------------------------------------------------------------------
%	MARGINS
%-------------------------------------------------------------------------------

\usepackage{geometry} % Required for adjusting page dimensions and margins

\geometry{
	paper=a4paper, % Paper size, change to letterpaper for US letter size
	top=2.5cm, % Top margin
	bottom=3cm, % Bottom margin
	left=2.5cm, % Left margin
	right=2.5cm, % Right margin
	headheight=14pt, % Header height
	footskip=1.5cm, % Space from the bottom margin to the baseline of the footer
	headsep=1.2cm, % Space from the top margin to the baseline of the header
	%showframe, % Uncomment to show how the type block is set on the page
}

%-------------------------------------------------------------------------------
%	FONTS
%-------------------------------------------------------------------------------

\usepackage[utf8]{inputenc}         % encodings
\usepackage[english,russian]{babel} % localization
\usepackage{indentfirst}            % for first line indent

%-------------------------------------------------------------------------------
%	BOXING NOTES
%-------------------------------------------------------------------------------

% Description:
%  Красивый блок примечания, который можно вставлять в середину главы.
%
% Usage:
% \begin{info}[optional title, defaults to "Info:"]
% 	contents
% \end{info}
%
% Notes:
%  Если вы хотите изменить черный цвет на другой, то отредактируйте команды
%  \fill и \draw.
%

\mdfdefinestyle{info}{%
	topline=false, bottomline=false,
	leftline=false, rightline=false,
	nobreak,
	singleextra={%
		\fill[black](P-|O)circle[radius=0.4em];
		\node at(P-|O){\color{white}\scriptsize\bfseries i};
		\draw[black, very thick](P-|O)++(0,-0.8em)--(O);%--(O-|P);
	}
}

% Define a custom environment for information
\newenvironment{info}[1][Примечание:]{ % Set the default title
	\medskip
	\begin{mdframed}[style=info]
		\noindent{\textbf{#1}}
}{
	\end{mdframed}
}

% Description:
%  Красивый блок с предупреждением, который можно вставлять в середину главы.
%
% Usage:
% \begin{warn}[optional title, defaults to "Warning:"]
%	Contents
% \end{warn}
%
% Notes:
%  Если вы хотите изменить черный цвет на другой, то отредактируйте команды
%  \fill и \draw.
%

\mdfdefinestyle{warning}{
	topline=false, bottomline=false,
	leftline=false, rightline=false,
	nobreak,
	singleextra={%
		\draw(P-|O)++(-0.5em,0)node(tmp1){};
		\draw(P-|O)++(0.5em,0)node(tmp2){};
		\fill[black,rotate around={45:(P-|O)}](tmp1)rectangle(tmp2);
		\node at(P-|O){\color{white}\scriptsize\bfseries !};
		\draw[very thick](P-|O)++(0,-1em)--(O);%--(O-|P);
	}
}

% Define a custom environment for warning text
\newenvironment{warn}[1][Внимание:]{ % Set the default warning"
	\medskip
	\begin{mdframed}[style=warning]
		\noindent{\textbf{#1}}
}{
	\end{mdframed}
}

% Description:
%  Красивый блок, в который можно вставлять тексты командной оболочки.
%
% Usage:
% \begin{commandline}
%	\begin{verbatim}
%		$ ls
%		
%		Applications	Desktop	...
%	\end{verbatim}
% \end{commandline}

\mdfdefinestyle{commandline}{
	leftmargin=10pt,
	rightmargin=10pt,
	innerleftmargin=15pt,
	middlelinecolor=black!50!white,
	middlelinewidth=2pt,
	frametitlerule=false,
	backgroundcolor=black!5!white,
	frametitle={Командная оболочка},
	frametitlefont={\normalfont\sffamily\color{white}\hspace{-1em}},
	frametitlebackgroundcolor=black!50!white,
	nobreak,
}

% Define a custom environment for command-line snapshots
\newenvironment{commandline}{
	\medskip
	\begin{mdframed}[style=commandline]
}{
	\end{mdframed}
	\medskip
}

% Description:
%  Красивый блок для разных нужд
%
% Usage:
% \begin{rblock}[optional title]
%	Contents
% \end{rblock}

\mdfdefinestyle{rblock}{
	innertopmargin=1.2\baselineskip,
	innerbottommargin=0.8\baselineskip,
	roundcorner=5pt,
	nobreak,
	singleextra={%
		\draw(P-|O)node[xshift=1em,anchor=west,fill=white,draw,rounded corners=5pt]{%
		\theTitle};
	},
}

\newenvironment{rblock}[1][\unskip]{
	\bigskip
	\newcommand{\theTitle}{#1}
	\begin{mdframed}[style=rblock]
}{
	\end{mdframed}
	\medskip
}
