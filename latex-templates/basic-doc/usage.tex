%%%%%%%%%%%%%%%%%%%%%%%%%%%%%%%%%%%%%%%%%%%%%%%%%%%%%%%%%%%%%%%%%%%%%%%%%%%%%%%%
% Basic document example
% Version: 1.0 (2019-04-29)
%
% Shell command: pdflatex --shell-escape usage.tex
%
% Author: Grigory Okhmak (ohmak88@yandex.ru)
% License:
% CC BY-NC-SA 3.0 (http://creativecommons.org/licenses/by-nc-sa/3.0/)
%
%%%%%%%%%%%%%%%%%%%%%%%%%%%%%%%%%%%%%%%%%%%%%%%%%%%%%%%%%%%%%%%%%%%%%%%%%%%%%%%%

\documentclass{scrartcl}
%-------------------------------------------------------------------------------
%	Commons
%-------------------------------------------------------------------------------
%%%%%%%%%%%%%%%%%%%%%%%%%%%%%%%%%%%%%%%%%%%%%%%%%%%%%%%%%%%%%%%%%%%%%%%%%%%%%%%%
% Basic elements
% Version: 1.0 (2019-04-29)
%
% Author: Grigory Okhmak (ohmak88@yandex.ru)
% License:
% CC BY-NC-SA 3.0 (http://creativecommons.org/licenses/by-nc-sa/3.0/)
%
%%%%%%%%%%%%%%%%%%%%%%%%%%%%%%%%%%%%%%%%%%%%%%%%%%%%%%%%%%%%%%%%%%%%%%%%%%%%%%%%

%-------------------------------------------------------------------------------
%	PACKAGES AND OTHER DOCUMENT CONFIGURATIONS
%-------------------------------------------------------------------------------

\usepackage{amsmath,amsfonts,stmaryrd,amssymb} % Math packages
\usepackage{enumerate} % Custom item numbers for enumerations
\usepackage[ruled]{algorithm2e} % Algorithms
\usepackage[framemethod=tikz]{mdframed} % Allows defining custom boxed/framed environments
\usepackage{listings} % File listings, with syntax highlighting
\lstset{
	basicstyle=\ttfamily, % Typeset listings in monospace font
}

%-------------------------------------------------------------------------------
%	MARGINS
%-------------------------------------------------------------------------------

\usepackage{geometry} % Required for adjusting page dimensions and margins

\geometry{
	paper=a4paper, % Paper size, change to letterpaper for US letter size
	top=2.5cm, % Top margin
	bottom=3cm, % Bottom margin
	left=2.5cm, % Left margin
	right=2.5cm, % Right margin
	headheight=14pt, % Header height
	footskip=1.5cm, % Space from the bottom margin to the baseline of the footer
	headsep=1.2cm, % Space from the top margin to the baseline of the header
	%showframe, % Uncomment to show how the type block is set on the page
}

%-------------------------------------------------------------------------------
%	FONTS
%-------------------------------------------------------------------------------

\usepackage[utf8]{inputenc}         % encodings
\usepackage[english,russian]{babel} % localization
\usepackage{indentfirst}            % for first line indent

%-------------------------------------------------------------------------------
%	BOXING NOTES
%-------------------------------------------------------------------------------

% Description:
%  Красивый блок примечания, который можно вставлять в середину главы.
%
% Usage:
% \begin{info}[optional title, defaults to "Info:"]
% 	contents
% \end{info}
%
% Notes:
%  Если вы хотите изменить черный цвет на другой, то отредактируйте команды
%  \fill и \draw.
%

\mdfdefinestyle{info}{%
	topline=false, bottomline=false,
	leftline=false, rightline=false,
	nobreak,
	singleextra={%
		\fill[black](P-|O)circle[radius=0.4em];
		\node at(P-|O){\color{white}\scriptsize\bfseries i};
		\draw[black, very thick](P-|O)++(0,-0.8em)--(O);%--(O-|P);
	}
}

% Define a custom environment for information
\newenvironment{info}[1][Примечание:]{ % Set the default title
	\medskip
	\begin{mdframed}[style=info]
		\noindent{\textbf{#1}}
}{
	\end{mdframed}
}

% Description:
%  Красивый блок с предупреждением, который можно вставлять в середину главы.
%
% Usage:
% \begin{warn}[optional title, defaults to "Warning:"]
%	Contents
% \end{warn}
%
% Notes:
%  Если вы хотите изменить черный цвет на другой, то отредактируйте команды
%  \fill и \draw.
%

\mdfdefinestyle{warning}{
	topline=false, bottomline=false,
	leftline=false, rightline=false,
	nobreak,
	singleextra={%
		\draw(P-|O)++(-0.5em,0)node(tmp1){};
		\draw(P-|O)++(0.5em,0)node(tmp2){};
		\fill[black,rotate around={45:(P-|O)}](tmp1)rectangle(tmp2);
		\node at(P-|O){\color{white}\scriptsize\bfseries !};
		\draw[very thick](P-|O)++(0,-1em)--(O);%--(O-|P);
	}
}

% Define a custom environment for warning text
\newenvironment{warn}[1][Внимание:]{ % Set the default warning"
	\medskip
	\begin{mdframed}[style=warning]
		\noindent{\textbf{#1}}
}{
	\end{mdframed}
}

% Description:
%  Красивый блок, в который можно вставлять тексты командной оболочки.
%
% Usage:
% \begin{commandline}
%	\begin{verbatim}
%		$ ls
%		
%		Applications	Desktop	...
%	\end{verbatim}
% \end{commandline}

\mdfdefinestyle{commandline}{
	leftmargin=10pt,
	rightmargin=10pt,
	innerleftmargin=15pt,
	middlelinecolor=black!50!white,
	middlelinewidth=2pt,
	frametitlerule=false,
	backgroundcolor=black!5!white,
	frametitle={Командная оболочка},
	frametitlefont={\normalfont\sffamily\color{white}\hspace{-1em}},
	frametitlebackgroundcolor=black!50!white,
	nobreak,
}

% Define a custom environment for command-line snapshots
\newenvironment{commandline}{
	\medskip
	\begin{mdframed}[style=commandline]
}{
	\end{mdframed}
	\medskip
}

% Description:
%  Красивый блок для разных нужд
%
% Usage:
% \begin{rblock}[optional title]
%	Contents
% \end{rblock}

\mdfdefinestyle{rblock}{
	innertopmargin=1.2\baselineskip,
	innerbottommargin=0.8\baselineskip,
	roundcorner=5pt,
	nobreak,
	singleextra={%
		\draw(P-|O)node[xshift=1em,anchor=west,fill=white,draw,rounded corners=5pt]{%
		\theTitle};
	},
}

\newenvironment{rblock}[1][\unskip]{
	\bigskip
	\newcommand{\theTitle}{#1}
	\begin{mdframed}[style=rblock]
}{
	\end{mdframed}
	\medskip
}
 % a4paper amsmath amsfonts stmaryrd amssymb
%-------------------------------------------------------------------------------
%	Addition styles
%-------------------------------------------------------------------------------
\usepackage{misccorr} % для оформления, принятого в Российской Федерации
\usepackage{graphicx} % для вставки картинок
\usepackage{minted}   % для оформления исходных кодов

%-------------------------------------------------------------------------------
%	Basic document
%-------------------------------------------------------------------------------

\begin{document}

%-------------------------------------------------------------------------------
%	Введение
%-------------------------------------------------------------------------------
\section*{Введение}
Далеко-далеко за словесными горами в стране гласных и согласных живут рыбные тексты.
Вдали от всех живут они в буквенных домах на берегу Семантика большого языкового океана.
Маленький ручеек Даль журчит по всей стране и обеспечивает ее всеми необходимыми\  
правилами.
\begin{info}
    Это блок примечания. Вы можете изменить текст примечания, если вы укажете второй\ 
    необязательный аргумент для этого блока.
\end{info}
Эта парадигматическая страна, в которой жаренные члены предложения залетают прямо в рот.
Даже всемогущая пунктуация не имеет власти над рыбными текстами, ведущими безорфографичный образ жизни.
Однажды одна маленькая строчка рыбного текста по имени Lorem ipsum решила выйти в\ 
большой мир грамматики. Великий Оксмокс предупреждал ее о злых запятых, диких знаках\ 
вопроса и коварных точках с запятой, но текст не дал сбить себя с толку. Он собрал\ 
семь своих заглавных букв, подпоясал инициал за пояс и пустился в дорогу.
\begin{warn}
    Это блок с предупреждением. Сюда вы можете выносить <<подводные камни>> к сказанному\ 
    в этой главе.
\end{warn}
Взобравшись на первую вершину курсивных гор, бросил он последний взгляд назад,\ 
на силуэт своего родного города Буквоград, на заголовок деревни Алфавит и на подзаголовок\ 
своего переулка Строчка. Грустный риторический вопрос скатился по его щеке и он\ 
продолжил свой путь. По дороге встретил текст рукопись. Она предупредила его:\ 
«В моей стране все переписывается по несколько раз. Единственное, что от меня\ 
осталось, это приставка «и». Возвращайся ты лучше в свою безопасную страну». Не\ 
послушавшись рукописи, наш текст продолжил свой путь. Вскоре ему повстречался коварный составитель.

%-------------------------------------------------------------------------------
%	Глава 1
%-------------------------------------------------------------------------------
\section{Заголовок первого уровня}
Далеко-далеко за словесными горами в стране гласных и согласных живут рыбные тексты.
Вдали от всех живут они в буквенных домах на берегу Семантика большого языкового океана.
Маленький ручеек Даль журчит по всей стране и обеспечивает ее всеми необходимыми\  
правилами.
\begin{rblock}[Необязательный заголовок]
    Содержимое блока.
\end{rblock}
Эта парадигматическая страна, в которой жаренные члены предложения залетают прямо в рот.
Даже всемогущая пунктуация не имеет власти над рыбными текстами, ведущими безорфографичный образ жизни.
Однажды одна маленькая строчка рыбного текста по имени Lorem ipsum решила выйти в\ 
большой мир грамматики. Великий Оксмокс предупреждал ее о злых запятых, диких знаках\ 
вопроса и коварных точках с запятой, но текст не дал сбить себя с толку. Он собрал\ 
семь своих заглавных букв, подпоясал инициал за пояс и пустился в дорогу.

Взобравшись на первую вершину курсивных гор, бросил он последний взгляд назад,\ 
на силуэт своего родного города Буквоград, на заголовок деревни Алфавит и на подзаголовок\ 
своего переулка Строчка. Грустный риторический вопрос скатился по его щеке и он\ 
продолжил свой путь. По дороге встретил текст рукопись. Она предупредила его:\ 
«В моей стране все переписывается по несколько раз. Единственное, что от меня\ 
осталось, это приставка «и». Возвращайся ты лучше в свою безопасную страну». Не\ 
послушавшись рукописи, наш текст продолжил свой путь. Вскоре ему повстречался коварный составитель.

%-------------------------------------------------------------------------------
%	Глава 1.1
%-------------------------------------------------------------------------------
\subsection{Заголовок второго уровня}
Далеко-далеко за словесными горами в стране гласных и согласных живут рыбные тексты.
Вдали от всех живут они в буквенных домах на берегу Семантика большого языкового океана.
Маленький ручеек Даль журчит по всей стране и обеспечивает ее всеми необходимыми\  
правилами.

Эта парадигматическая страна, в которой жаренные члены предложения залетают прямо в рот.
Даже всемогущая пунктуация не имеет власти над рыбными текстами, ведущими безорфографичный образ жизни.
Однажды одна маленькая строчка рыбного текста по имени Lorem ipsum решила выйти в\ 
большой мир грамматики. Великий Оксмокс предупреждал ее о злых запятых, диких знаках\ 
вопроса и коварных точках с запятой, но текст не дал сбить себя с толку. Он собрал\ 
семь своих заглавных букв, подпоясал инициал за пояс и пустился в дорогу.

Взобравшись на первую вершину курсивных гор, бросил он последний взгляд назад,\ 
на силуэт своего родного города Буквоград, на заголовок деревни Алфавит и на подзаголовок\ 
своего переулка Строчка. Грустный риторический вопрос скатился по его щеке и он\ 
продолжил свой путь. По дороге встретил текст рукопись. Она предупредила его:\ 
«В моей стране все переписывается по несколько раз. Единственное, что от меня\ 
осталось, это приставка «и». Возвращайся ты лучше в свою безопасную страну». Не\ 
послушавшись рукописи, наш текст продолжил свой путь. Вскоре ему повстречался коварный составитель.

%-------------------------------------------------------------------------------
%	Глава 1.1.1
%-------------------------------------------------------------------------------
\subsubsection{Заголовок третьего уровня}
Далеко-далеко за словесными горами в стране гласных и согласных живут рыбные тексты.
Вдали от всех живут они в буквенных домах на берегу Семантика большого языкового океана.
Маленький ручеек Даль журчит по всей стране и обеспечивает ее всеми необходимыми\  
правилами.

Эта парадигматическая страна, в которой жаренные члены предложения залетают прямо в рот.
Даже всемогущая пунктуация не имеет власти над рыбными текстами, ведущими безорфографичный образ жизни.
Однажды одна маленькая строчка рыбного текста по имени Lorem ipsum решила выйти в\ 
большой мир грамматики. Великий Оксмокс предупреждал ее о злых запятых, диких знаках\ 
вопроса и коварных точках с запятой, но текст не дал сбить себя с толку. Он собрал\ 
семь своих заглавных букв, подпоясал инициал за пояс и пустился в дорогу.

Взобравшись на первую вершину курсивных гор, бросил он последний взгляд назад,\ 
на силуэт своего родного города Буквоград, на заголовок деревни Алфавит и на подзаголовок\ 
своего переулка Строчка. Грустный риторический вопрос скатился по его щеке и он\ 
продолжил свой путь. По дороге встретил текст рукопись. Она предупредила его:\ 
«В моей стране все переписывается по несколько раз. Единственное, что от меня\ 
осталось, это приставка «и». Возвращайся ты лучше в свою безопасную страну». Не\ 
послушавшись рукописи, наш текст продолжил свой путь. Вскоре ему повстречался коварный составитель.

%-------------------------------------------------------------------------------
%	Глава 2
%-------------------------------------------------------------------------------
\section{Оформление исходных текстов программ}
Для вставки исходного текста кода в командной оболочке можно использовать следующий\ 
блок.
\begin{commandline}
    \begin{verbatim}
        $ cd ~
        $ ./say_hello.sh

        Hello
    \end{verbatim}
\end{commandline}
Для вставки текстов исходных кодов можно использовать пакет minted. Убедитесь, что\ 
этот пакет установлен.

Перед установкой убедитесь, что установлен Python-пакет Pygments.
\begin{minted}{bash}
    sudo pip install Pygments
\end{minted}
Пример кода, вставленного через minted.
\begin{minted}[linenos]{c++}
#include <iostream>

int main(int argc, char* argv[])
{
    std::cout << "Hello, World!" << std::endl;
    return 0;
}
\end{minted}

\end{document}